\chapter*{Project Proposal}
\begin{flushleft}
    The world we live in today is one that is increasingly data-centric.   Along with our ability to generate more data must also come an increase in our capacity to make sense of that data.  As organizations attempt to scale their environments, bottlenecks often arise within the underlying relational database. Some organizations use this as a reason to migrate to a cloud or hybrid environment, while others, whether from necessity or preference, remain in an on-premises environment. In this paper, we have sought to explore a new type of hybrid model that will facilitate the vertical scaling of local RDBMS systems while avoiding the more cumbersome remedies of adding hardware or offloading logic into application layers. The design underpinning this analysis was constructed to allow examination of the feasibility and performance of utilizing cloud computational resources to augment the throughput of local relational database systems while avoiding the need for additional hardware and minimizing disruption of the existing code base.  
    \\

    While startups and personal endeavors are typically small and agile, it is the larger enterprises that struggle against inertia and must come to grips with the long tailed transitions that would come along with cloud adoption. Through this project our team will explore reasons to migrate to cloud while also try not to get caught up in the hype. While everyone sees the cents per service unit , are there actual savings in the long run? With this research we want to explore why's and ifs to be considered before moving  conventional systems to cloud. 

\end{flushleft}
  
\pagebreak

\section*{Why this Project?}
The potential benefits from that would come along with the ability to scale a local database in a way that is flexible and ultra-low cost are obvious: low barrier to entry for small organizations, mitigation of security concerns around cloud storage, inherent cost savings.




\section*{Solution Specifics}
Databases provide a prime use case for on-premises private and hybrid cloud models. For our research we will use on-prem relational database and explore hybrid models that can either re-use any of the on-prem resources for storage and compute with the flexibility to be scalable to cloud on need basis or look for a potential model that may be cost effective and  provide cloud advantages while moving storage and compute entirely on cloud. To test this team will explore use of APIs and serverless compute services offered by cloud providers.